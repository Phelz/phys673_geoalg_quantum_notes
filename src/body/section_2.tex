
\section{The Core of Geometric Algebra}
\subsection{Vectors, Bivectors, and Multivectors}

I assume here that the reader is familiar with scalars and vectors, shown in FIG XX. In GA, we built to higher-dimensional (called higher-grade) objects such as bivectors, shown in FIG YY. A bivector is the directed area, given by the parallelogram formed by two vectors. The mathematical operation that forms bivectors from vectors is called the outer product--often referred to as the wedge product. This operation is anti-symmetric so that 
\begin{equation} \label{eq:outer_product}
    \large\boxed{\vec{u} \ \wedge \ \vec{v} = - \ \vec{v} \ \wedge \ \vec{u} },
\end{equation}
for any two vectors $\vec{u}$ and $\vec{v}$. The anti-symmetry is reminiscent of the cross product, and we shall see the connection shorty. For now, I would like to indicate that equation \ref{eq:outer_product} is important, and I shall be using it frequently. 
\\ \\
As an aside, one could construct even higher grade objects, oriented volumes or hypervolumes, called trivectors and multivectors, respectively; see Figure ZZZ. However, I shall not need more than to show, without rigour, how these objects come about, namely via successive applications of the outer product. For inSGAnce, taking the outer product orthonormal basis $\{ \vec{\vec{e_i}} \} \vee i \in \{ 1, 2,3\} $, corresponding to $ \{ \hat{i}, \hat{j}, \hat{k} \}$:
\begin{equation} \label{eq:pseudoscalar_wedge_definition}
   \vec{\vec{e_1}} \wedge \vec{\vec{e_2}} \wedge \vec{\vec{e_3}} \ = \mathbb{I}  
\end{equation}
yields the cube of unit volume in 3D space shown in FIG ZZZ, where $\mathbb{I}$ is suspiciously called the \textbf{pseudoscalar}--a very important quantity we shall encounter time and again in our discussions of GA.
\\ \\
Now you have a geometric idea of the outer product--no more a mysterious or purely algebraic operation. The outer product is a grade-raising operation, as we saw in the examples for bivectors and trivectors. For multivectors, you can imagine more of the same.

\subsection{The Geometric Product}

At the core of any algebra is an operation akin to multiplication. In linear algebra, we multiply matrices via specific rules. In GA, the rule is simple, for any vectors $\vec{u}$ and $\vec{v}$, their geometric product is defined via
\begin{equation}
    \large \boxed{\vec{u} \vec{v} = \vec{u} \cdot \vec{v} \ + \vec{u} \ \wedge  \ \vec{v} },
\end{equation}
where $\vec{u} \cdot \vec{v}$ resembles the dot product, but I shall call it the inner product from now on; the reason for the terminology shall become apparent. This choice of nomenclature does not alter the mathematics--for now, and the reader should interpret this term in the usual sense. Namely, the inner product of two vectors is a scalar, a measure of how two vectors are aligned in space, scaled by their lengths. Thus, the inner product term produces a grade-0 object in this case. 
\\ \\
As for the geometric product, while slightly unnerving at first glance--given the indoctrination of linear algebra typical of any physicist--can be interpreted using an analogy with complex numbers, which we never question. Much like we add the complex numbers: 
$$ (a + ib) + (c -id) = (a+c) + (c-d)i , $$
one can think of same-grade components in a multi-vector adding together. Apples add with apples and oranges with oranges.
\\ \\
We need not contend too much with this idea, at least for the moment, since, most of the time the geometric product reduces to either the inner or outer product term. As an example, consider again the orthonormal basis for $\mathbb{R}^3$. The product 
\begin{equation}
   \vec{\vec{e_1}} \vec{\vec{e_2}} = \vec{\vec{e_1}} \wedge \vec{\vec{e_2}},  
\end{equation}
because $\vec{\vec{e_1}} \cdot \vec{\vec{e_2}} = 0$, since the basis vectors are orthogonal. This applies to any of the basis vectors. On the other hand,
\begin{equation}
   \vec{\vec{e_1}} \vec{\vec{e_1}} = \vec{\vec{e_1}} \cdot \vec{\vec{e_1}} = 1, 
\end{equation}
since the vectors are normalized. This simplification scheme will be our bread and butter in what is to come.