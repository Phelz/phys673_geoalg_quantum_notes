\documentclass{article}
%Math Stuff
\usepackage{amsmath} % many display options for math modes and such (VERY IMPORTANT)
\usepackage{systeme}
\usepackage{physics} % useful math symbols and macros (VERY IMPORTANT)
\usepackage{amssymb} % heck ton of math symbols and fonts (VERY IMPORTANT)
\usepackage{amsfonts} % Additional fonts, math symbols, and options for existing fonts (IMPORTANT)
\usepackage{mathtools} % lots of cool mathtools oh wait that is the name (IMPORTANT)
\usepackage{siunitx} % allows you to quickly define units within math mode (IMPORTANT)
\usepackage{array} % allows you to write piece-wise functions, matrices, and other cool things (IMPORTANT)
\usepackage{txfonts} % defines times new roman as default text font and provides supporting math symbols
\usepackage{braket} % allows the use of detailed Dirac braket notation
\usepackage{halloweenmath} % spooky
\usepackage{empheq} %Extension of amsmath for boxing equations or answers
\usepackage{cancel}


%Formatting Stuff
\usepackage[utf8]{inputenc} % the typesetting rules
\usepackage{geometry} % flexible and easy interface to change page dimensions
\usepackage{graphicx} % provides additional options for figures
\usepackage{float} % allows you to tell latex no really I want the figure HERE (with H)
\usepackage{color} % provides coloring capabilities to everything
\usepackage{fancyhdr} % making fancy headers (obviously)
\usepackage{changepage} % allows to change page formatting in the middle of a document (rather than the same throughout) 
\usepackage{enumitem} % better control over enumerate and itemize
\usepackage[labelfont=bf]{caption} % altering options for captions
\usepackage{sidecap} % allows sideways captions for figures and tables
\usepackage{soul,xcolor} % can enhance striking, underlining, and highlighting (in non-math mode)
\usepackage{subcaption}


%Stuff for Tables
\usepackage{booktabs} % allows weird lines within tables
\usepackage{multirow}
\usepackage{multicol} % allows for multiple columns and rows that collapse into single columns
\usepackage{longtable} % allows a table to run over multiple pages
\usepackage{colortbl} % colourful tables!

%Stuff for Citing
\usepackage[english]{babel}
\usepackage{url} % allows options for url links
\usepackage[colorlinks, citecolor = black, urlcolor = black, linkcolor = black]{hyperref} %references (citation) options and hyperlink integration


%Elijah Adams Created Commands
\newcommand{\Q}{\mathbb{Q}} %Mathematical sets
\newcommand{\R}{\mathbb{R}}
\newcommand{\C}{\mathbb{C}}
\newcommand{\Z}{\mathbb{Z}} 
\newcommand{\Curl}{\vec{\nabla}\times} % Good looking curl and div and laplace (because physics package ones look like shit)
\newcommand{\Div}{\vec{\nabla}\cdot}
\newcommand{\laplace}{\vec{\nabla}^2}
\newcommand{\yas}{\frac{1}{4\pi\epsilon_0}} % Useful for E&M stuff
\newcommand{\mas}{\frac{\mu_0}{4\pi}}
\newcommand{\ppdv}[2]{\frac{\partial^2 #1}{\partial #2^2}} % Twice partial derivatives

\usepackage{pdfpages}
\usepackage{svg}
\usepackage{fancyhdr}
\usepackage{mathrsfs}
\usepackage{amsfonts}
\usepackage{xcolor}
\usepackage{hyperref, nameref, xurl} % links/ labels
\sisetup{multi-part-units=single}

\newcommand{\deriv}[2]{\frac{d#1}{d#2}}
\newcommand{\pderiv}[1]{\frac{\partial}{\partial#1}}
\newcommand{\pderivf}[2]{\frac{\partial#1}{\partial#2}}
\newcommand{\pderivw}[3]{\left(\frac{\partial #1}{\partial #2}\right)_{#3}}
\renewcommand{\exp}[1]{e^{#1}}
\renewcommand{\lambda}{\lambdaup}
\newcommand{\mean}[1]{\left<#1\right>}
\newcommand{\func}[2]{#1\left(#2\right)}
\DeclareMathOperator\arctanh{tanh^{-1}}
%\renewcommand{\vec}[1]{\mathbf{#1}}
\renewcommand{\vec}[1]{\boldsymbol{#1}}
\newcommand{\nvec}[1]{\hat{\mathbf{#1}}}
\newcommand{\Lagr}{\mathscr{L}}
\newcommand{\Fdual}{\mathscr{F}}

\newcommand{\cse}{{{T_\Fdual}^\alpha}_\beta}

\newcommand{\LF}{\Lagr_\Fdual}

\newcommand{\metric}{\eta_{\alpha\beta}}

\newcommand{\SubItem}[1]{
    {\setlength\itemindent{15pt} \item[-] #1}
}

%\counterwithin{figure}{section}
%\counterwithin{table}{section}
%\counterwithin{equation}{section}

\linespread{1.25}

\pagestyle{fancy}
\fancyhf{}
\rhead{F. Ghaly}
\lhead{PHYS 673 Lecture Notes}
\rfoot{Page \thepage}


\begin{document}
\section{Introduction}

ADD 5 REFERENCES!!!

This lecture is intended to introduce Geometric Algebra (GA). My goal is to show that the complementarity of the spin $\frac{1}{2}$ system arises not from something physical, but rather the even subalgebra, hidden within the GA of spacetime. This should be sufficient to entice the reader to ask the question: 
\\ \\
"What is the connection between spin-1/2 systems and geometry?"
\\ \\
In approaching this goal, I shall dive deeper into concepts commonly overlooked in any quantum mechanics course. These include:
\begin{enumerate}
    \item the cross product, outer product, and their relationship to duality,
    \item the Levi-Cevita tensor (only to hint at more fundamental relationships),
    \item the inner product and its relationship to the metric tensor,
    \item the even subalgebra of spacetime and its relationship to Pauli matrices--there out called Pauli vectors,
    \item and finally, deriving the commutator relationships of the even subalgebra, giving a brief outlook on its relationship to spin-1/2 systems.
\end{enumerate}



\section{The Core of Geometric Algebra}
\subsection{Vectors, Bivectors, and Multivectors}

I assume here that the reader is familiar with scalars and vectors, shown in FIG XX. In GA, we built to higher-dimensional (called higher-grade) objects such as bivectors, shown in FIG YY. A bivector is the directed area, given by the parallelogram formed by two vectors. The mathematical operation that forms bivectors from vectors is called the outer product--often referred to as the wedge product. This operation is anti-symmetric so that 
\begin{equation} \label{eq:outer_product}
    \large\boxed{\vec{u} \ \wedge \ \vec{v} = - \ \vec{v} \ \wedge \ \vec{u} },
\end{equation}
for any two vectors $\vec{u}$ and $\vec{v}$. The anti-symmetry is reminiscent of the cross product, and we shall see the connection shorty. For now, I would like to indicate that equation \ref{eq:outer_product} is important, and I shall be using it frequently. 
\\ \\
As an aside, one could construct even higher grade objects, oriented volumes or hypervolumes, called trivectors and multivectors, respectively; see Figure ZZZ. However, I shall not need more than to show, without rigour, how these objects come about, namely via successive applications of the outer product. For inSGAnce, taking the outer product orthonormal basis $\{ \vec{\vec{e_i}} \} \vee i \in \{ 1, 2,3\} $, corresponding to $ \{ \hat{i}, \hat{j}, \hat{k} \}$:
\begin{equation} \label{eq:pseudoscalar_wedge_definition}
   \vec{\vec{e_1}} \wedge \vec{\vec{e_2}} \wedge \vec{\vec{e_3}} \ = \mathbb{I}  
\end{equation}
yields the cube of unit volume in 3D space shown in FIG ZZZ, where $\mathbb{I}$ is suspiciously called the \textbf{pseudoscalar}--a very important quantity we shall encounter time and again in our discussions of GA.
\\ \\
Now you have a geometric idea of the outer product--no more a mysterious or purely algebraic operation. The outer product is a grade-raising operation, as we saw in the examples for bivectors and trivectors. For multivectors, you can imagine more of the same.

\subsection{The Geometric Product}

At the core of any algebra is an operation akin to multiplication. In linear algebra, we multiply matrices via specific rules. In GA, the rule is simple, for any vectors $\vec{u}$ and $\vec{v}$, their geometric product is defined via
\begin{equation}
    \large \boxed{\vec{u} \vec{v} = \vec{u} \cdot \vec{v} \ + \vec{u} \ \wedge  \ \vec{v} },
\end{equation}
where $\vec{u} \cdot \vec{v}$ resembles the dot product, but I shall call it the inner product from now on; the reason for the terminology shall become apparent. This choice of nomenclature does not alter the mathematics--for now, and the reader should interpret this term in the usual sense. Namely, the inner product of two vectors is a scalar, a measure of how two vectors are aligned in space, scaled by their lengths. Thus, the inner product term produces a grade-0 object in this case. 
\\ \\
As for the geometric product, while slightly unnerving at first glance--given the indoctrination of linear algebra typical of any physicist--can be interpreted using an analogy with complex numbers, which we never question. Much like we add the complex numbers: 
$$ (a + ib) + (c -id) = (a+c) + (c-d)i , $$
one can think of same-grade components in a multi-vector adding together. Apples add with apples and oranges with oranges.
\\ \\
We need not contend too much with this idea, at least for the moment, since, most of the time the geometric product reduces to either the inner or outer product term. As an example, consider again the orthonormal basis for $\mathbb{R}^3$. The product 
\begin{equation}
   \vec{\vec{e_1}} \vec{\vec{e_2}} = \vec{\vec{e_1}} \wedge \vec{\vec{e_2}},  
\end{equation}
because $\vec{\vec{e_1}} \cdot \vec{\vec{e_2}} = 0$, since the basis vectors are orthogonal. This applies to any of the basis vectors. On the other hand,
\begin{equation}
   \vec{\vec{e_1}} \vec{\vec{e_1}} = \vec{\vec{e_1}} \cdot \vec{\vec{e_1}} = 1, 
\end{equation}
since the vectors are normalized. This simplification scheme will be our bread and butter in what is to come.

\section{The Cross Product, Levi-Cevita, and Duality}
\subsection{The Cross Product}

Now that we have the basics, I would like to anchor our discussion in something familiar: the cross product. I would like us to examine a not-so-famous expression for the cross product between two vectors $ \vec{A} $ and $\vec{B} $:
\begin{equation} \label{eq:cross_product_levi_cevita}
    \vec{A} \cross \vec{B} = \epsilon_{ijk} a_i b_j \vec{e_k},
\end{equation}
where I have used Einstein summation notation, since 
\begin{align*}
    \vec{A} &= \sum_i a_i \vec{ \vec{e_i} }, \ &\ \vec{B} &= \sum_j b_j\vec{ \vec{e_j} }.
\end{align*}
You might have previously seen equation \ref{eq:cross_product_levi_cevita}. If not, I intentionally leave it to the reader to see that it matches the linear algebra definition computationally. The nominal definition of the Levi-Cevita is described only in terms of cyclic and anti-cyclic permutations of $1, 2, 3$:
\begin{equation}
    \label{eq:levicivita_nominal_def}
    \epsilon_{ijk} = 
    \begin{cases} 
          1 & \text{cyclic permutations of 1,2,3} \\
          -1 & \text{anti-cyclic permutations of 1,2,3} \\
          0 & \text{repeating indices}
       \end{cases}
\end{equation}
The appearance of the Levi-Cevita in equation \ref{eq:cross_product_levi_cevita} implicitly implies that for a right-handed orthonormal basis set
\begin{equation}
    \label{eq:cross_product_basis_vectors}
     \vec{\vec{e_i}} \cross \vec{\vec{e_j}} = \epsilon_{ijk} \vec{\vec{e_k}} .
\end{equation}
I shall intentionally refrain from commenting about the appearance of the Levi-Cevita in equation \ref{eq:cross_product_levi_cevita} (and by extension \ref{eq:cross_product_basis_vectors}). Indeed, this appearance is no coincidence, with deep geometrical implications. However, I shall not give away the entire answer here. Rather, I plan to entice the reader(s) to explore the answer for themselves. In the following, I shall give some of the necessary ingredients. A full geometric treatment of Levi-Cevita, however, requires a lecture of its own. In that spirit, I shall quote the GA definition of Levi-Cevita [REFERENCE HERE!!!!!] without proof:
\begin{equation}
    \label{eq:levicivita_ga_def}
    \epsilon_{ijk} = \vec{\vec{e_i}} \wedge \vec{\vec{e_j}} \wedge \vec{\vec{e_k}} \mathbb{I}^\dagger,
\end{equation}
where $\mathbb{I}^\dagger = \vec{\vec{e_3}} \wedge \vec{\vec{e_2}} \wedge \vec{\vec{e_1}} = \vec{\vec{e_3}} \ \vec{\vec{e_2}}  \ \vec{\vec{e_1}}  $, the reverse of equation \ref{eq:pseudoscalar_wedge_definition}.
We can verify that this definition is valid, by comparing with the more familiar definition, where
\begin{enumerate}
    \item For the case where any two indices repeat:
    $$ \epsilon_{112} = \vec{e_1} \wedge \vec{e_1} \wedge \vec{e_2} \mathbb{I}^\dagger $$
    where, $\vec{e_1} \wedge \vec{e_1} = 0$, yielding $ \epsilon_{112} = 0 $. This result generalizes, since $ \vec{e_i} \wedge \vec{e_i} = 0$:
    $$ \epsilon_{iij} =  \vec{e_i} \wedge \vec{e_i}  \wedge \vec{e_j} \mathbb{I}^\dagger = 0.$$

    \item For cyclic permutations:
    $$ \epsilon_{123} = \vec{e_1} \wedge \vec{e_2} \wedge \vec{e_3} \mathbb{I}^\dagger = \vec{e_1} \vec{e_2} \color{teal} \vec{e_3} \vec{e_3} \color{black}\vec{e_2} \vec{e_1} = 1, $$
    since $ \vec{e_i} \vec{e_i} = \vec{e_i} \cdot \vec{e_i} = 1$, and we see that the reversion of the pseudoscalar, $\mathbb{I}$, leads to the cyclic property of the Levi-Civita tensor.

    \item As for anti-cyclic permutations, our orthogonal basis set ensures anti-commutation:
    \begin{align*}
        \epsilon_{132} &= \vec{e_1} \wedge \vec{e_3} \wedge \vec{e_2} \vec{e_3} \vec{e_2} \vec{e_1} \\
        &= \vec{e_1} \vec{e_3} \vec{e_2} \vec{e_3} \vec{e_2} \vec{e_1} \\
        &= - \vec{e_1} \vec{e_2} \color{teal}\vec{e_3} \vec{e_3} \color{black} \vec{e_2} \vec{e_1} \\
        &= - \epsilon_{123} = -1,
    \end{align*}
    which ensures that the Levi-Civita is completely anti-symmetric under the swapping of any two indices.
    
\end{enumerate}
Therefore, we recover the definition of equation \eqref{eq:levicivita_nominal_def} using that of equation \eqref{eq:levicivita_ga_def}.

\subsection{The Outer Product}
We shall return to the cross product shortly. For now, I would like to make a correspondence between the two. So let us compute the outer product of $\vec{A} $ and $\vec{B} $, which can be simply expanded--all you need to know is associativity can distributivity:
\begin{align*}
    \vec{A} \wedge \vec{B}  &= (a_1 \vec{e_1} + a_2 \vec{e_2} + a_3 \vec{e_3}) \wedge (b_1 \vec{e_1} + b_2 \vec{e_2} + b_3 \vec{e_3}) \\
                &= a_1 \vec{e_1} \wedge b_1 \vec{e_1} + a_1 \vec{e_1} \wedge b_2 \vec{e_2} + a_1 \vec{e_1} \wedge b_3 \vec{e_3} \\
                &+ a_2 \vec{e_2} \wedge b_1 \vec{e_1} + a_2 \vec{e_2} \wedge b_2 \vec{e_2} + a_2 \vec{e_2} \wedge b_3 \vec{e_3} \\
                &+ a_3 \vec{e_3} \wedge b_1 \vec{e_1} + a_3 \vec{e_3} \wedge b_2 \vec{e_2} + a_3 \vec{e_3} \wedge b_3 \vec{e_3}.
\end{align*}
Notice that terms where two parallel bases are wedged vanish ($\vec{e_i} \wedge \vec{e_i} = 0$), and we are left with parts that are totally anti-symmetric. Thus the outer product reads
\begin{align*}
    \vec{A}  \wedge \vec{B}   &= (a_1 b_2 - a_2 b_1) \ \vec{e_1} \wedge \vec{e_2} \\
                &+ (a_2 b_3 - a_3 b_2) \ \vec{e_2} \wedge \vec{e_3} \\
                &+ (a_1 b_3 - a_3 b_1 ) \ \vec{e_1} \wedge \vec{e_3} 
\end{align*}
Focusing for now on the first term $ (a_1 b_2 - a_2 b_1) \ \vec{e_1} \wedge \vec{e_2}$, the wedge can be manipulated as such:
\begin{align*}
    \vec{e_1} \wedge \vec{e_2} = \vec{e_1} \vec{e_2} \vec{e_3} \vec{e_3} =  \vec{e_3} \vec{e_1} \vec{e_2} \vec{e_3},
\end{align*}
and we can utilize the fact that $\epsilon_{123} = 1$ along with equation \eqref{eq:cross_product_basis_vectors} to write:
\begin{align*}
    \vec{e_1} \wedge \vec{e_2} &= \epsilon_{123} \vec{e_3} \vec{e_1} \vec{e_2} \vec{e_3} \\
                    &= \vec{e_1} \cross \vec{e_2} \mathbb{I}.
\end{align*}
We can use this to rewrite the wedge product as 
\begin{align*}
    \vec{A}  \wedge \vec{B}   &= (a_1 b_2 - a_2 b_1) \  (\vec{e_1} \cross \vec{e_2}) \mathbb{I} \\
                &+ (a_2 b_3 - a_3 b_2) \  (\vec{e_2} \cross \vec{e_3}) \mathbb{I} \\
                &+ (a_1 b_3 - a_3 b_1 ) \ (\vec{e_1} \cross \vec{e_3}) \mathbb{I},
\end{align*}
or, 
\begin{align*}
    \vec{A}  \wedge \vec{B}   &= (a_1 b_2 - a_2 b_1) \  (\vec{e_3}) \mathbb{I} \\
                &+ (a_2 b_3 - a_3 b_2) \  (-\vec{e_2}) \mathbb{I} \\
                &+ (a_1 b_3 - a_3 b_1 ) \ (\vec{e_1}) \mathbb{I}.
\end{align*}
Here, we see the cross product makes an appearance! First, notice that the minus sign that was given by the determinant of the matrix defining the cross product is now built into this algebra. If you are reading this, this is a hint of something deeper! Second, the outer product can now be expressed in a compact form
\begin{equation}
    \label{eq:cross_product_outer_product}
    \large\boxed{\vec{A}  \wedge \vec{B}  = (\vec{A}  \cross \vec{B} ) \mathbb{I}}
\end{equation}


\subsection{Duality} \label{sec:duality}
The result of equation \eqref{eq:cross_product_outer_product} is profound and has a clear geometric interpretation. As we already know, the vector $ \vec{A}  \cross \vec{B}  $ is perpendicular to the plane spanned by $ \vec{A}  \wedge \vec{B} $. What about in 2D? What is perpendicular to $\vec{e_1}$? Well, there is only $\vec{e_2}$. Coincidently, 
$$ \vec{e_2} = \vec{e_2} \ 1 = \vec{e_2}  \ \vec{e_1} \vec{e_1}= \vec{e_1}  \ \vec{e_1} \vec{e_2} = \vec{e_1}  \ \mathbb{I}.$$
But what is perpendicular to $\vec{e_2}$?
\begin{align*}
\vec{e_2} &= \vec{e_1}  \ \mathbb{I} \\
\vec{e_2} \mathbb{I} &= \vec{e_1}  \ \mathbb{I} \mathbb{I} \\
\vec{e_2} \mathbb{I} &= \vec{e_1}  \ (\mathbb{I})^2=  \vec{e_1}  \ (\vec{e_1} \vec{e_2} \vec{e_3})^2 \\
\vec{e_2} \mathbb{I} &=  \vec{e_1}  \ (\vec{e_1} \vec{e_2} \vec{e_3} \vec{e_1} \vec{e_2} \vec{e_3} )\\
\vec{e_2} \mathbb{I} &=  \vec{e_1}  \ (\vec{e_2} \vec{e_3} \color{teal}{\vec{e_1} \vec{e_1}} \color{black} \vec{e_2} \vec{e_3} )\\
\vec{e_2} \mathbb{I} &=  \vec{e_1}  \ (-  \vec{e_3} \color{teal}{\vec{e_2} \vec{e_2}} \color{black} \vec{e_3} )
\end{align*}
\begin{equation} \label{eq:dual_3d_test_1}
\large\boxed{\vec{e_2} \mathbb{I} =  -\vec{e_1}}
\end{equation}
or,
\begin{align*}
 -\vec{e_1} &=  \vec{e_2} \mathbb{I} \\
 -\vec{e_1} &=\vec{e_2} \vec{e_1} \vec{e_2} = - \vec{e_2} \vec{e_2} \vec{e_1}
\end{align*}
\begin{equation} \label{eq:dual_3d_test_2}
\large\boxed{\vec{e_1} = \vec{e_2} \mathbb{I}^\dagger}.
\end{equation}
Woah! Ignoring the miraculous, very suspicious $\mathbb{I^2} = -1$ for a second, I tackle the more immediate task at hand: does this orthogonality by mere multiplication with the pseudoscalar, $\mathbb{I}$, generalize to higher dimensions? Well, it has to! Let us say, for some reason, we want to pick out an element of the space that is perpendicular to all other elements, then we certainly can. Here is how we do it. We can construct from our space a vector $\vec{e^j}$ perpendicular to all vectors $\vec{e_i}$ where $i \neq j$,
\begin{equation} \label{eq:duality_definition}
    \large\boxed{\vec{e_i} \cdot \vec{e^j} = \delta_i^j},
\end{equation}
 where $ \delta_i^j$ is the usual Kronecker delta. This can be achieved if we consider the unit volume spanned by our basis vectors (the pseudoscalar of our algebra) 
\begin{equation}
    \mathbb{I} \equiv \vec{e_1} \wedge \vec{e_2} \wedge \vec{e_3} ... \wedge \vec{e_n},
\end{equation}
for some $n$-dimensional space, spanned by $\vec{e_n}$. Then we can find $\vec{e^j}$ via
\begin{equation}
    \label{eq:dual_generator}
    \vec{e^j} = (-1)^{j-1} \vec{e_1} \wedge \vec{e_2} \wedge ... \hat{e}_j \wedge ... e_{n} \mathbb{I}^\dagger,
\end{equation}
where $\hat{e}_j$ in equation \eqref{eq:dual_generator} just implies that term is missing from the product. I shall ignore the $(-1)^{j-1}$ term for now--sufices to say it is related to the handedness of our set. 
\\ \\ 
For now let us consider what equation \eqref{eq:dual_generator} tells us. It says, to find a vector $\vec{e^j}$ perpendicular to all other basis elements, you first wedge all those elements to construct a hypervolume, but one which does NOT include the original vector $\vec{e_j}$. \footnote{Notice the subscript vs. superscript notation used to denote the difference between a vector vs. covector, respectively. Fancy names do not mean much, but we need some nomenclature to differentiate between the two.} After wedging all the elements, you multiply with the reversed pseudoscalar, $\mathbb{I}^\dagger$, (similar to what we did above) to pick out an element perpendicular to all those wedged elements. 
\\ \\
Let us see if this works. As an example, consider $\vec{e_2}$, but now in 3D space:
\begin{align*}
    \vec{e_2} \cdot \vec{e^2} &= \vec{e_2} \cdot ((-1)^1 \vec{e_1} \wedge \vec{e_3} \mathbb{I}^\dagger ) \\
                    &= \vec{e_2} \cdot (-\vec{e_1} \vec{e_3} \mathbb{I}^\dagger )\\
                    &= \vec{e_2} \cdot (-\vec{e_2} \vec{e_2} \vec{e_1} \vec{e_3} \mathbb{I}^\dagger )\\
                    &= \vec{e_2} \cdot (\vec{e_2} \mathbb{I} \mathbb{I}^\dagger)\\
                    &= \vec{e_2} \cdot \vec{e_2} = 1.
\end{align*}
where $\mathbb{I} \mathbb{I}^\dagger = \vec{e_1} \vec{e_2} \color{teal} \vec{e_3} \vec{e_3} \color{black} \vec{e_2} \vec{e_1} = 1$. Now, one might argue that this result is redundant. We tried to look for an element much like $\vec{e_2}$ in the space, in the sense that it is perpendicular to $\vec{e_1}$ and $\vec{e_3}$. However, we ended up right back at $\vec{e_2}$. However, although this result implies that 
\begin{equation*}
    \vec{e_i} = \vec{e^i},
\end{equation*} 
one cannot take that to be the case for all vector spaces. As we shall see in the next section, the metric of the space dictates the rules that relate a vector to its covector--or, put in more familiar terms, a vector to its dual. Now the term "dual" is no longer mysterious and has a grounded, geometrical meaning.


\section{An Algebra within an Algebra}

\subsection{Spacetime Geometric Algebra}
In the last section, I introduced the idea of duality, and said that it is not always the case that the basis vector and their duals are identical--a somewhat redundant result. To illustrate that this is not always the case, let us consider a slightly more sophisticated algebra, the Spacetime Geometric Algebra (SGA), which inherits the 4D Minkowski metric from special relativity. I assume the reader is already familiar with that tensor. I will be using the mostly negative version, the one with the signature: $(+, -, -, -)$.
\\ \\ 
The orthonormal basis of SGA consists of one timelike vector $\gamma_0$ and three spacelike vectors $\gamma_1$, $\gamma_2$, and $\gamma_3$. Like with normal Euclidean space, the basis vectors follow two important rules. The first identifies orthogonality:
\begin{equation}
   \large\boxed{ \vec{\gamma_\alpha} \vec{\gamma_\beta} = -\vec{\gamma_\beta}\vec{\gamma_\alpha}},
\end{equation}
while the second, and most important, dictates the operation of the inner product via:
\begin{equation} \label{eq:metric_dot_product}
    \large\boxed{\vec{\gamma_\alpha}\vec{\gamma_\alpha} = \vec{\gamma_\alpha} \cdot \vec{\gamma_\alpha} = \large{ \eta_{\alpha\alpha}}},
\end{equation} 
since the outer product $\vec{\gamma_\alpha} \wedge \vec{\gamma_\alpha}$ vanishes. I cannot overstress how fundamental equation \eqref{eq:metric_dot_product} is. The metric defines the structure of the geometry. By defining the norm, it also implicitly defines the rules of dual transformations--the reason as to why, I shall leave to the reader. As for how, consider the spacelike basis vector resembling $\hat{y}$, namely $\vec{\gamma_2}$, and its norm: \footnote{By norm, I refer to the inner product; length would be the square root of that.}
\begin{equation} \label{eq:dot_product_gamma_2}
\vec{\gamma_2} \cdot \vec{\gamma_2} = \eta_{22} = -1.
\end{equation}
Because of equation \eqref{eq:duality_definition}, which defines duality, and equation \eqref{eq:dot_product_gamma_2} above, we must have that:
$$    \vec{\gamma_2} \cdot \vec{\gamma^2} = 1 \Longrightarrow \vec{\gamma_2} \cdot (-\vec{\gamma_2} )= - \vec{\gamma_2} \cdot \vec{\gamma_2} =  -(-1) = 1, $$
meaning,
\begin{equation}
    \large\boxed{\vec{\gamma_i} = - \vec{\gamma^i}}. 
\end{equation}
But that only holds for $i \in \{ 1, 2,3\}$. For the timelike vector, 
\begin{equation}
    \large\boxed{\vec{\gamma_0} =  \vec{\gamma^0}}.  
\end{equation}
It is a good exercise to see why the latter must be true using a similar argument to the one described above. Conversely, if our metric was of the mostly plus signature, $(-, +, +, +)$, we would have:
\begin{align*}
    \vec{\gamma^0} = - \vec{\gamma_0 } && \vec{\gamma^i} = \vec{ \gamma_i}.
\end{align*}
Therefore, the inner product, defined by the metric tensor, sets the rules of the algebra. As we say in the language of tensors: "the metric tensor is used to raise or lower indices.":
\begin{align*}
\vec{\gamma_\alpha} = \vec{\gamma^\beta } \ \eta_{\alpha \beta} && 
\vec{\gamma^\alpha} = \vec{\gamma_\beta } \ \eta^{\alpha \beta} 
\end{align*}


\subsection{The Even Subalgebra}
The last section should have been indicative of the complexity of SGA, given its metric. Another aspect of this intricacy arises when we consider the amount of grade-2 elements we can generate using our basis set $\{ \vec{\gamma_0}, \vec{\gamma_2}, \vec{\gamma_1}, \vec{\gamma_3} \}$. Considering that scalars form via the products $\vec{\gamma_0} \vec{\gamma_0} = 1$ and $\vec{\gamma_i}\vec{\gamma_i}= -1 $, the only generators for bivectors are 
\begin{table}[H]
\centering
\begin{tabular}{ccccc}
$\{\vec{\gamma_i}\vec{\gamma_0}\}$ & and & $\{\vec{\gamma_i} \vec{\gamma_j}\},$ 
\end{tabular}
\end{table}
each of which generates 3 independent bivectors (given their anti-commutativity), for a total of 6 bivectors in our spacetime geom{etric algebra. I would like to focus on the bivector algebra formed by the set $ \{ \vec{\sigma_i} \} \equiv \{\vec{\gamma_i}\vec{\gamma_0}\}$, from which you can generate scalars via
\begin{equation} \label{eq:pauli_unitary}
    (\vec{\sigma_i})^2 = \vec{\sigma_i} \vec{\sigma_i}  =  \vec{\gamma_i} \vec{\gamma_0} \vec{\gamma_i} \vec{\gamma_0} = - \vec{\gamma_i} \vec{\gamma_i}  \color{teal} \vec{\gamma_0} \vec{\gamma_0} \color{black} = 1,
\end{equation}
and the pseudoscalar via,
\begin{align*}
    \mathbb{I} &=  \vec{\sigma_1} \vec{\sigma_2} \vec{\sigma_3} \\
    &= \vec{\gamma_1} \vec{\gamma_0} \vec{\gamma_2} \vec{\gamma_0} \vec{\gamma_3} \vec{\gamma_0} \\
    &=-\vec{\gamma_0} \vec{\gamma_1} \vec{\gamma_2} \vec{\gamma_0} \vec{\gamma_3} \vec{\gamma_0} \\
    &= \vec{\gamma_0} \vec{\gamma_1} \vec{\gamma_2} \vec{\gamma_3} \color{teal}\vec{\gamma_0} \vec{\gamma_0},
\end{align*}
\begin{equation}
    \large \boxed{ \mathbb{I} =  \vec{\sigma_1} \vec{\sigma_2} \vec{\sigma_3}=\vec{\gamma_0} \vec{\gamma_1} \vec{\gamma_2} \vec{\gamma_3}   }.
\end{equation}
This means that the subalgebra and SGA share the same pseudoscalar.
\\ \\ 
Borrowing on from equation \eqref{eq:levicivita_ga_def}, within this subalgebra, the Levi-Civita symbol is defined in terms of subalgebra's pseudoscalar as
\begin{align*}
    \epsilon_{ijk} &=  \vec{\sigma_i} \wedge \vec{\sigma_j} \wedge \vec{\sigma_k} \mathbb{I}^\dagger \\
                    &= \vec{\sigma_i} \vec{\sigma_j} \vec{\sigma_k} \vec{\sigma_3} \vec{\sigma_2} \vec{\sigma_1} \\
    \epsilon_{ijk} \mathbb{I} &= \vec{\sigma_i} \vec{\sigma_j} \vec{\sigma_k} \vec{\sigma_3}\vec{\sigma_2} \color{teal} \vec{\sigma_1} \vec{\sigma_1} \color{black} \vec{\sigma_2} \vec{\sigma_3} \\
    \epsilon_{ijk} \mathbb{I} &= \vec{\sigma_i} \vec{\sigma_j} \vec{\sigma_k} \\
    \epsilon_{ijk} \mathbb{I} \vec{\sigma_k} &= \vec{\sigma_i} \vec{\sigma_j} \vec{\sigma_k}  \vec{\sigma_k} 
\end{align*}
\begin{equation} \label{eq:pauli_product}
    \large \boxed{ \vec{\sigma_i} \vec{\sigma_j} = \epsilon_{ijk} \mathbb{I} \vec{\sigma_k}  }
\end{equation}
This is no trivial result. Because the Levi-Civita is totally anti-symmetric, equation \eqref{eq:pauli_product} implies
\begin{equation}
    \large \boxed{\vec{\sigma_i} \vec{\sigma_j} = - \vec{\sigma_j} \vec{\sigma_i}},
\end{equation}
which implies
\begin{equation} \label{eq:pauli_commutator}
    \vec{\sigma_i} \vec{\sigma_j} = \frac{1}{2} \left(  \vec{\sigma_i} \vec{\sigma_j} - \vec{\sigma_j} \vec{\sigma_i}  \right) = \frac{1}{2} \left[ \vec{\sigma_j}, \vec{\sigma_j} \right].
\end{equation}
Finally, combining equations \eqref{eq:pauli_product} with \eqref{eq:pauli_commutator}, we arrive at the commutator relationship governing the famous Pauli matrices:
\begin{equation}
    \large \boxed{ \left[ \vec{\sigma_j}, \vec{\sigma_j} \right] = 2 \epsilon_{ijk} \mathbb{I} \vec{\sigma_k}  }.
\end{equation}
Since the product of any of the $\vec{\sigma_i} \vec{\sigma_j} $ reduces to $1$ if $i=j$ (see equation \eqref{eq:pauli_unitary}) and results in the third orthogonal bivector otherwise (see equation \eqref{eq:outer_product}), the algebra is summarized by
\begin{equation}
    \large \boxed{ \vec{\sigma_i} \vec{\sigma_j} = \delta_{ij} +  \epsilon_{ijk} \mathbb{I} \vec{\sigma_k}  }.
\end{equation}
Looks familiar?... Indeed, this is the algebra Pauli introduced to describe quantum spin! However, I would like us to pause here and ponder a little, since, up to this point, I had made no mention of quantum mechanics, nor did I introduce any physics, observables, measurements, etc. We arrived here purely via the geometric algebra of spacetime. Does this mean quantum spin is inextricably linked to geometry? While, I cannot fully answer this question, nor did I find anyone to claim they can in the literature, I can certainly start to pierce the veil--show you a glimpse of the geometry of quantum spin. 



\section{Geometry of Spin}

\subsection{Rotations}
In the last section, I discussed the Pauli algebra, and how it arises from the bivector algebra within SGA. I would like to expand on this here, by first making a correspondence with a familiar operator. Consider
\begin{equation} \label{eq:euler_quantum}
	\large  \exp{i \hat{\sigma_z} \theta / 2} = \cos(\theta/2) + i \hat{\sigma_z} \sin(\theta/2),
\end{equation}
where $\hat{\sigma_z} $ is the quantum operator (sigma matrix) analog to our $\vec{\sigma_3}$. Equation \eqref{eq:euler_quantum} is relatively known in quantum mechanics, and can be verified by expanding the exponential function. The proof is simple and can be found in many textbooks. I would like to make the argument that the GA analog also holds, namely
\begin{equation} \label{eq:euler_ga}
	\large  \exp{ \mathbb{I} \vec{\sigma_z} \theta / 2} = \cos(\theta/2) + \mathbb{I} \vec{\sigma_z} \sin(\theta/2).
\end{equation}
Instead of going through the math, I would like to remind the reader of the fundamental result we stumbled upon while deriving the dual in equation \eqref{eq:dual_3d_test_1}:
\begin{equation} \label{eq:imaginary_pseudoscalar}
	\mathbb{I}  \mathbb{I} = \mathbb{I} ^2 = -1,
\end{equation}
which I swept under rug right momentarily. This result alludes to a hidden complex structure embedded within GA. Equation \eqref{eq:imaginary_pseudoscalar} holds not only for the pseudoscalar of 3D space which was used in deriving equation \eqref{eq:dual_3d_test_1}, but also for the 4D Minkowski space (as one quickly check), and, by extension, for the even subalgebra. This result is precisely why GA
\\ \\
Additionally, I have shown in the previous section that the Pauli vectors are unitary (see equation \eqref{eq:pauli_unitary}). Hence, we have all the ingredients necessary to assert that if equation \eqref{eq:euler_quantum} holds equation \eqref{eq:euler_ga} must also hold.
\\ \\
Finally, using equation \eqref{eq:pauli_product}, I would like to rewrite  $\mathbb{I} \vec{\sigma_3}$.
$$ \vec{\sigma_i} \vec{\sigma_j} = \epsilon_{ijk} \mathbb{I} \vec{\sigma_k} = \epsilon_{ij3} \mathbb{I} \vec{\sigma_3},$$
which implies,
$$ \mathbb{I} \vec{\sigma_3 } = \vec{\sigma_1 } \vec{\sigma_2 }. $$
Rewriting equation \eqref{eq:euler_ga}, we have
\begin{equation} \label{eq:euler_ga_rewritten}
	\large  \exp{ \mathbb{I} \vec{\sigma_z} \theta / 2} = \cos(\theta/2) + \vec{\sigma_1} \vec{\sigma_2} \sin(\theta/2).
\end{equation}
Here, I would like to examine the sandwich product
\begin{equation} \label{eq:sandwich_product}
	\large  \exp{ -\mathbb{I} \vec{\sigma_z}  \theta / 2} \ \vec{v}  \
	\exp{ \mathbb{I} \vec{\sigma_z} \theta / 2},
\end{equation}
for some vector $\vec{v}$ in our subalgebra. To illustrate, let us use $\vec{v} =  \vec{\sigma_1}$:
% \begin{align*}
% 	\exp{ -\mathbb{I} \vec{\sigma_z} \theta / 2} \ \vec{\sigma_1}  \
% 	\exp{ \mathbb{I} \vec{\sigma_z} \theta / 2} & = \left(  \cos(\frac{\theta}{2}) - \vec{\sigma_1} \vec{\sigma_2} \sin(\frac{\theta}{2}) \right) \vec{\sigma_1} \left(  \cos(\frac{\theta}{2}) + \vec  {\sigma_1} \vec{\sigma_2} \sin(\frac{\theta}{2}) \right) \\
% 	                                            & = \left(  \vec{\sigma_1} \cos(\frac{\theta}{2}) + \vec{\sigma_2}  \sin(\frac{\theta}{2}) \right) \left(  \cos(\frac{\theta}{2}) + \vec{\sigma_1} \vec{\sigma_2} \sin(\frac{\theta}{2}) \right) \\
% 	                                            & =  \vec{\sigma_1} \cos[2](\frac{\theta}{2})  + \vec{\sigma_2}  \sin(\frac{\theta}{2}) \cos(\frac{\theta}{2}) + \color{teal}\vec{\sigma_1} \vec{\sigma_1} \color{black} \vec{\sigma_2} \cos(\frac{\theta}{2})  \sin(\frac{\theta}{2}) + \vec{\sigma_2} \vec{\sigma_1} \vec{\sigma_2} \sin(\frac{\theta}{2})  \sin(\frac{\theta}{2}) \\
% 	                                            & =  \vec{\sigma_1} \cos[2](\frac{\theta}{2})+ \vec{\sigma_2}  \sin(\frac{\theta}{2}) \cos(\frac{\theta}{2}) + \vec{\sigma_2} \cos(\frac{\theta}{2})  \sin(\frac{\theta}{2}) - \color{teal} \vec{\sigma_2} \vec{\sigma_2} \color{black} \vec{\sigma_1} \sin[2](\frac{\theta}{2})   \\
% 	                                            & =  \vec{\sigma_1} \cos[2](\frac{\theta}{2})+ 2 \vec{\sigma_2}  \sin(\frac{\theta}{2}) \cos(\frac{\theta}{2}) - \vec{\sigma_1} \sin[2](\frac{\theta}{2})
% \end{align*}
% \begin{align*}
% 	\exp{ -\mathbb{I} \vec{\sigma_z} \theta / 2} \ \vec{\sigma_1}  \
% 	\exp{ \mathbb{I} \vec{\sigma_z} \theta / 2}  = \left(  \cos(\frac{\theta}{2}) - \vec{\sigma_1} \vec{\sigma_2} \sin(\frac{\theta}{2}) \right) \vec{\sigma_1} \left(  \cos(\frac{\theta}{2}) + \vec  {\sigma_1} \vec{\sigma_2} \sin(\frac{\theta}{2}) \right) \\
%                                                  =   \cos(\frac{\theta}{2}) \vec{\sigma_1} \cos(\frac{\theta}{2}) 
%                                                   - \vec{\sigma_1} \vec{\sigma_2} \sin(\frac{\theta}{2})  \ \vec{\sigma_1} \cos(\frac{\theta}{2})  \\
%                                                   &\quad + \cos(\frac{\theta}{2}) \vec{\sigma_1}  \vec{\sigma_1} \vec{\sigma_2} \sin(\frac{\theta}{2}) 
%                                                   -  \vec  {\sigma_1} \vec{\sigma_2} \sin(\frac{\theta}{2})  \vec  {\sigma_1} \vec{\sigma_2} \sin(\frac{\theta}{2}) \\
%                                                   % --------------------
%                                                  =  \vec{\sigma_1}   \cos[2](\frac{\theta}{2}) 
%                                                   - \vec{\sigma_1} \vec{\sigma_2}  \ \vec{\sigma_1} \sin(\frac{\theta}{2})  \cos(\frac{\theta}{2})  \\
%                                                   &\quad + \color{teal} \vec{\sigma_1}  \vec{\sigma_1}  \color{black} \vec{\sigma_2} \sin(\frac{\theta}{2})  \cos(\frac{\theta}{2})
%                                                   -  \vec  {\sigma_1} \vec{\sigma_2} \vec{\sigma_1} \vec{\sigma_2}    \sin[2](\frac{\theta}{2}) \\
%                                                   % --------------------
%                                                  =  \vec{\sigma_1}   \cos[2](\frac{\theta}{2}) 
%                                                   + \vec{\sigma_2} \color{teal} \vec{\sigma_1}  \ \vec{\sigma_1} \color{black} \sin(\frac{\theta}{2})  \cos(\frac{\theta}{2})  \\
%                                                   &\quad + \vec{\sigma_2} \sin(\frac{\theta}{2})  \cos(\frac{\theta}{2})
%                                                   -  (\vec  {\sigma_1} \vec{\sigma_2} \vec{\sigma_1} \vec{\sigma_2})    \sin[2](\frac{\theta}{2}) \\
% \end{align*}

$$ 	\exp{ -\mathbb{I} \vec{\sigma_z} \theta / 2} \ \vec{\sigma_1}  \
\exp{ \mathbb{I} \vec{\sigma_z} \theta / 2}  = \left(  \cos(\frac{\theta}{2}) - \vec{\sigma_1} \vec{\sigma_2} \sin(\frac{\theta}{2}) \right) \vec{\sigma_1} \left(  \cos(\frac{\theta}{2}) + \vec  {\sigma_1} \vec{\sigma_2} \sin(\frac{\theta}{2}) \right) $$
\begin{align*} 
                                                 &=   \cos(\frac{\theta}{2}) \vec{\sigma_1} \cos(\frac{\theta}{2}) 
                                                  - \vec{\sigma_1} \vec{\sigma_2} \sin(\frac{\theta}{2})  \ \vec{\sigma_1} \cos(\frac{\theta}{2})
                                                  + \cos(\frac{\theta}{2}) \vec{\sigma_1}  \vec{\sigma_1} \vec{\sigma_2} \sin(\frac{\theta}{2}) 
                                                  -  \vec  {\sigma_1} \vec{\sigma_2} \sin(\frac{\theta}{2}) \vec{\sigma_1}  \vec{\sigma_1} \vec{\sigma_2} \sin(\frac{\theta}{2}) \\
                                                  % --------------------
                                                 &=  \vec{\sigma_1}   \cos[2](\frac{\theta}{2}) 
                                                  - \vec{\sigma_1} \vec{\sigma_2}  \ \vec{\sigma_1} \sin(\frac{\theta}{2})  \cos(\frac{\theta}{2}) 
                                                  + \color{teal} \vec{\sigma_1}  \vec{\sigma_1}  \color{black} \vec{\sigma_2} \sin(\frac{\theta}{2})  \cos(\frac{\theta}{2})
                                                  -  \vec  {\sigma_1} \vec{\sigma_2} \vec{\sigma_1} \vec{\sigma_1} \vec{\sigma_2}    \sin[2](\frac{\theta}{2}) \\
                                                  % --------------------
                                                 &=  \vec{\sigma_1}   \cos[2](\frac{\theta}{2}) 
                                                  + \vec{\sigma_2} \color{teal} \vec{\sigma_1}  \ \vec{\sigma_1} \color{black} \sin(\frac{\theta}{2})  \cos(\frac{\theta}{2})
                                                  + \vec{\sigma_2} \sin(\frac{\theta}{2})  \cos(\frac{\theta}{2})
                                                  -  \vec  {\sigma_1} \vec{\sigma_1} \vec{\sigma_2} \vec{\sigma_1} \vec{\sigma_2}    \sin[2](\frac{\theta}{2}) \\
                                                  % --------------------
                                                 &=  \vec{\sigma_1}   \cos[2](\frac{\theta}{2}) 
                                                 + 2 \vec{\sigma_2} \sin(\frac{\theta}{2})  \cos(\frac{\theta}{2})
                                                 - \vec{\sigma_1}  (\vec  {\sigma_1} \vec{\sigma_2} )^2    \sin[2](\frac{\theta}{2})
\end{align*}
or,
\begin{equation} \label{eq:sigma_x_rotated}
     \large \exp{ -\mathbb{I} \vec{\sigma_z} \theta / 2} \ \vec{\sigma_1}  \
     \exp{ \mathbb{I} \vec{\sigma_z} \theta / 2} =  \vec{\sigma_1} \cos(\theta) + \vec{\sigma_2} \sin(\theta),
\end{equation}
where, in the last step, I made use of the half-angle trigonometric identities and that $ (\vec{\sigma_1} \vec{\sigma_2} )^2 = \vec{\sigma_1} \vec{\sigma_2} \vec{\sigma_1} \vec{\sigma_2} = - \vec{\sigma_1} \color{teal} \vec{\sigma_2} \vec{\sigma_2} \color{black} \vec{\sigma_1} = -1 $, again, behaving like the imaginary unit. The latter is no coincidence. In fact, the Pauli algebra is isomorphic to the algebra of quaternions, where 
\begin{align*}
    \vec{\sigma_1} \mathbb{I} &\sim i &
    -\vec{\sigma_2} \mathbb{I} & \sim j &
    \vec{\sigma_3} \mathbb{I} &\sim k.
\end{align*}    
Hence, the quaternion algebra, fundamentally related to spatial rotations, is also a subalgebra of the SGA, and the imaginary unit $i$ is represented by the bivector $\vec{\sigma_1} \vec{\sigma_2}$. Equation \eqref{eq:sigma_x_rotated} describes a rotation of the vector $\vec{\sigma_1}$ about the $\vec{\sigma_3}$ axis by an angle $\theta$. Now we can visualize the rotation as being brought about by the action of the exponential of the bivector $\vec{\sigma_1} \vec{\sigma_2}$, which is a rotation operator in the plane defined by the two vectors $\vec{\sigma_1}$ and $\vec{\sigma_2}$. The sense of rotation would then be opposite if we use the bivector $-\vec{\sigma_1} \vec{\sigma_2}= \vec{\sigma_2} \vec{\sigma_1}$--a purely geometric intuition! 
\\ \\ 
One can check that this operation is indeed a rotation. For instance, substituting $\theta = \pi/2$ in equation \eqref{eq:sigma_x_rotated} gives
$$ \large \exp{ -\mathbb{I} \vec{\sigma_z} \pi / 4} \ \vec{\sigma_1}  \ \exp{ \mathbb{I} \vec{\sigma_z} \pi / 4} =  \vec{\sigma_2}. $$
Substituting $\theta = \pi/4$ gives
$$ \large \exp{ -\mathbb{I} \vec{\sigma_z} \pi / 8} \ \vec{\sigma_1}  \ \exp{ \mathbb{I} \vec{\sigma_z} \pi / 8} =  \frac{1}{\sqrt{2}} \left( \vec{\sigma_1} + \vec{\sigma_2} \right). $$
Indeed, we are moving around the unit circle in the plane defined by $\vec{\sigma_1}$ and $\vec{\sigma_2}$! Thus, using these rotations, we can construct the unit sphere, commonly referred to as the Bloch sphere; see figure ZZZ. In the next section, I shall explore how this sphere relates to the mathematical description of the qubit.

\subsection{The Reimann Sphere}


















% But notice that the product $\vec{\sigma_i} \vec{\sigma_i} = 1$:
% $$ \vec{\sigma_i} \vec{\sigma_i} = \vec{\gamma_i}\vec{\gamma_0} \vec{\gamma_i}\vec{\gamma_0} = - \vec{\gamma_i}\vec{\gamma_i}\vec{\gamma_0} \vec{\gamma_0}  = - (-1) = 1. $$
% This property is profound, particularly because we know that the product of these bivectors can only have two elements, a scalar and a bivector:
% \begin{align}
%     \label{eq:geomprod-sigmaij}
%     \vec{\sigma_i} \vec{\sigma_j} &=  \left<\vec{\sigma_i} \vec{\sigma_j}\right>_0 + \left<\vec{\sigma_i} \vec{\sigma_j}\right>_2
% \end{align}
% Since the only other choice for a bivector product within this subalgebra is $\vec{\sigma_i} \vec{\sigma_j} = -\vec{\gamma_i}\vec{\gamma_j}$ is a bivector, we know what the grade-0 projection must be the dot product, given by
% \begin{align}
%     \boxed{ \left<\vec{\sigma_i} \vec{\sigma_j}\right>_0 \equiv \vec{\sigma_i} \cdot \vec{\sigma_j} = \delta_i^j. }
% \end{align}
% Continuing with our derivation for equation \eqref{eq:21}, we get
% \begin{align*}
%     \epsilon_{ijk} \mathbf{I} &= \vec{\sigma_i} \vec{\sigma_j} \vec{\sigma_k} \vec{\sigma_3}\vec{\sigma_2} \vec{\sigma_1} \vec{\sigma_1} \vec{\sigma_2} \vec{\sigma_3} \\
%     &=  \vec{\sigma_i} \vec{\sigma_j} \vec{\sigma_k} \\
%     &=  \vec{\gamma_i}\vec{\gamma_0} \vec{\gamma_j} \vec{\gamma_0} \gamma_k \vec{\gamma_0} \\ 
%     &=  - \vec{\gamma_i}\vec{\gamma_0} \vec{\gamma_j}  \gamma_k 
% \end{align*}
% However, this subalgebra shares the same pseudoscalar as that of spacetime:
% \begin{equation}
%     \mathbf{I} = \vec{\sigma_1} \vec{\sigma_2} \vec{\sigma_3}= \gamma_1 \vec{\gamma_0} \gamma_2 \vec{\gamma_0} \gamma_3 \vec{\gamma_0} = \vec{\gamma_0} \gamma_1 \gamma_2 \gamma_3 = I.
% \end{equation}
% Therefore, we end up with equation \eqref{eq:21}: $ \vec{\gamma_i}\vec{\gamma_0} \vec{\gamma_j} \gamma_k = -  \epsilon_{ijk} I$. One way to rewrite this result is to use $\epsilon_{ijk} I = \vec{\sigma_i} \vec{\sigma_j} \vec{\sigma_k} $ and write
% \begin{equation}
%     \label{eq:levicivita-sigmas}
%      \vec{\sigma_i} \vec{\sigma_j}  = \epsilon_{ijk} I \vec{\sigma_k}.
% \end{equation}
% This gives us a way to map the product of two bivectors to a third bivector, hence we obtain the grade-2 element of this subalgebra- the cross product, by
% \begin{equation}
%     \boxed{ \left<\vec{\sigma_i} \vec{\sigma_j}\right>_2 \equiv \vec{\sigma_i} \cross \vec{\sigma_j}  = \epsilon_{ijk}  \vec{\sigma_k}}.
% \end{equation}
% The result of equation \eqref{eq:levicivita-sigmas} implies that the product $\vec{\sigma_i} \vec{\sigma_j}$ is anti-commutative due to the presence of the totally anti-symmetric Levi-Civita symbol, and therefore 
% $$ \vec{\sigma_i} \vec{\sigma_j} - \vec{\sigma_i} \vec{\sigma_j} = 2 \vec{\sigma_i} \vec{\sigma_j}. $$
% This leads to the cross product being defined in terms of the commutator relation
% \begin{align*}
%     \vec{\sigma_i} \cross \vec{\sigma_j} &=  \epsilon_{ijk} \vec{\sigma_k} \\
%                             &=  - \vec{\sigma_i} \vec{\sigma_j} I \\
%                             &= - \frac{1}{2} \left ( \vec{\sigma_i} \vec{\sigma_j} - \vec{\sigma_j} \vec{\sigma_i}  \right) I
% \end{align*}
% \begin{equation}
%     \boxed{\vec{\sigma_i} \cross \vec{\sigma_j} =  - \comm{\vec{\sigma_i}}{\vec{\sigma_j}} \ I},
% \end{equation}
% where the commutator is defined by $ \comm{\vec{\sigma_i}}{\vec{\sigma_j}} \equiv \frac{1}{2} \left ( \vec{\sigma_i} \vec{\sigma_j} - \vec{\sigma_j} \vec{\sigma_i}  \right) $. Additionally, this anti-commutative property corroborates our result for the inner product, since 
% $$ \vec{\sigma_i} \vec{\sigma_j} + \vec{\sigma_i} \vec{\sigma_j} = 0, $$
% while
% $$ \vec{\sigma_i} \vec{\sigma_i} + \vec{\sigma_i} \vec{\sigma_i} = 2. $$
% The inner product can now be redefined in terms of geometric products as
% \begin{equation}
%     \boxed{\vec{\sigma_i} \cdot \vec{\sigma_j} =  \frac{1}{2} \left ( \vec{\sigma_i} \vec{\sigma_j} + \vec{\sigma_i} \vec{\sigma_j}  \right)} 
% \end{equation}
% We can also use this anti-commutative property to show that the pesudoscalar squares to $-1$:
% \begin{equation}
%     I^2 = \vec{\sigma_1} \vec{\sigma_2} \vec{\sigma_3}\vec{\sigma_1} \vec{\sigma_2} \vec{\sigma_3}= \vec{\sigma_1} \vec{\sigma_2} \vec{\sigma_3}\vec{\sigma_3}\vec{\sigma_1} \vec{\sigma_2}  = \vec{\sigma_1} \vec{\sigma_2} \vec{\sigma_1} \vec{\sigma_2}  = -1
% \end{equation}
% This result alludes to the hidden complex structure within the geometric algebra of spacetime, and is another reason why the pesudoscalar behaves like the imaginary unit, $i$. Finally, we can use this and equation \eqref{eq:geomprod-sigmaij}, which reads $ - \vec{\gamma_i}\vec{\gamma_0} \vec{\gamma_j}  \gamma_k  = \epsilon_{ijk} I $, to write 
% \begin{align*}
%     - \vec{\gamma_i} \vec{\gamma_j} \vec{\sigma_k}  &= \epsilon_{ijk} I  \\
%     - \vec{\gamma_i} \vec{\gamma_j} \vec{\sigma_k}  I &= \epsilon_{ijk} I^2 \\
%     \vec{\sigma_k}  I &= \vec{\gamma_j} \vec{\gamma_i}\epsilon_{ijk} 
% \end{align*}
% or,
% \begin
% {equation}
%     I \vec{\sigma_k} =  - \epsilon_{ijk}   \vec{\gamma_i}\vec{\gamma_j} =  \epsilon_{ijk}   \sigma_i \vec{\sigma_j} .
% \end{equation}
% This last result showcases the dual relationship between the bivector elements of this subalgebra. Since the geometric product can be broken down as 
% $$ \sigma_i \vec{\sigma_j} = \sigma_i \cdot  \vec{\sigma_j} + \sigma_i \wedge \vec{\sigma_j} \,$$
% we can write
% \begin{align*}
%      \sigma_i \vec{\sigma_j} &=  \sigma_i \wedge \vec{\sigma_j} \\
%      - \sigma_i \vec{\sigma_j}  I &= -  \sigma_i \wedge \vec{\sigma_j}  I,
% \end{align*}
% or,
% \begin{equation}
%     \boxed{ \sigma_i \wedge \vec{\sigma_j} = \sigma_i \cross \vec{\sigma_j} \ I. }
% \end{equation}
% We therefore obtain the familiar dual relationship between the cross and outer products. \\

% At this point, in addition to duality, we have derived the two fundamental mathematical operations of the familiar 3D space, found within this subalgebra. The inner product reducing to a delta function implies we are dealing with a mutually orthogonal set. Thus we have all the mathematical structure needed for dealing with 3D space, all encapsulated within the spacetime algebra and we refer to this subalgebra as the Even Subalgebra.


% \subsection{Spacetime Bivectors}


% The minus sign hints at a relationship that requires the Levi-Civita tensor, as with all the other spaces (2D and 3D Euclidean space) explored thus far. We shall derive the exact relationship later on. For now, consider the geometric product of two bivectors. Given our generators, we can only have combinations of products of $\{\vec{\gamma_i}\vec{\gamma_0}\}$ and $\{\vec{\gamma_i}\vec{\gamma_j}\},$ and thus are limited to certain grades for our bivector product. 
% \begin{enumerate}
%     \item Grade 0 (Scalar):

%     When we have products of $\{\vec{\gamma_i}\vec{\gamma_0}\}$ or $\{\vec{\gamma_i}\vec{\gamma_j}\}$, the result must be a scalar. 
%         \SubItem{Products of $\{\vec{\gamma_i}\vec{\gamma_0}\}$}
%             $$ \vec{\gamma_i}\vec{\gamma_0} \vec{\gamma_i}\vec{\gamma_0} = - \vec{\gamma_i}\vec{\gamma_i}\vec{\gamma_0} \vec{\gamma_0} = 1$$
    
%         \SubItem{Products of $\{\vec{\gamma_i}\vec{\gamma_j}\}$}
%             $$ \vec{\gamma_i}\vec{\gamma_j} \vec{\gamma_i}\vec{\gamma_j} = - \vec{\gamma_i}\vec{\gamma_i}\vec{\gamma_j} \vec{\gamma_j} = -1$$
            
%     \item Grade 2 (Bivector):

%     When we have products of $\{\vec{\gamma_i}\vec{\gamma_j}\}$ with $\{\vec{\gamma_j} \gamma_k\}$ or $\{\vec{\gamma_i}\vec{\gamma_0}\}$ with $\{\vec{\gamma_i}\vec{\gamma_j}\}$, the result must be a bivector.

%         \SubItem{Products of $\{\vec{\gamma_i}\vec{\gamma_j}\}$  with $\{\vec{\gamma_j} \gamma_k\}$}:
%             $$ \vec{\gamma_i}\vec{\gamma_j} \vec{\gamma_j} \gamma_k = - \vec{\gamma_i}\gamma_k $$    
            
%         \SubItem{Products of $\{\vec{\gamma_i}\vec{\gamma_0}\} $ with $\{\vec{\gamma_i}\vec{\gamma_j}\}$}:
%             $$ \vec{\gamma_i}\vec{\gamma_0} \vec{\gamma_i}\vec{\gamma_j} = - \vec{\gamma_i}\vec{\gamma_i}\vec{\gamma_0} \vec{\gamma_j} = \vec{\gamma_0} \vec{\gamma_j} $$    
            
%     \item Grade 4 (Pseudoscalar):

%     When we take products of $\{\vec{\gamma_i}\vec{\gamma_0}\}$ with $\{\vec{\gamma_j} \gamma_k\}$.
%     \begin{equation}
%         \label{eq:21}
%         \vec{\gamma_i}\vec{\gamma_0} \vec{\gamma_j} \gamma_k = -  \epsilon_{ijk} I.    
%     \end{equation}
    
% \end{enumerate}
% That last result might not be immediately obvious, although it is clear that the product will be of grade 4 and therefore must be a signature multiple of the pesudoscalar. Notice that if we are dealing with general bivectors constructed from $\sigma_i = \vec{\gamma_i}\vec{\gamma_0}$, they cannot have grade-4 elements. We shall focus on such bivectors for now and derive the aforementioned result of equation \eqref{eq:21} in the next section.

% \subsection{The Inner Product}

% If you're familiar with matrices, you might have heard that vectors are column matrices while co-vectors are row matrices. Well, think of the co-vector $\vec{u}$ that maps $\vec{v} = \vec{e}_3$ to $\vec{0}$:

% \[
% \begin{bmatrix}
%     u_1 && u_2 && u_3
% \end{bmatrix}
% \begin{bmatrix}
%     0 \\
%     0 \\
%     1
% \end{bmatrix}
% =
% 0.
% \]
% I'm sure you agree that a row matrix multiplied with a column matrix gives us a scalar. Infact, this is the familiar dot product! So this equation amounts to the question: what vector is perpindicular to $\vec{e}_3$? And if you don't immediatly see that, we can simply crunch through the linear algebra to see that this corresponds to the plane $\vec{\vec{e_3}} = 0$, scaled by some factor $u_3$
% $$u_3 \vec{\vec{e_3} } = 0 $$
%  All vectors in this plane are perpindicular to $\vec{\vec{e_3}}$. 

%  As a last remark before I end this section, it is worth noting that this duality results further generalizes to the case of a non-orthogonal basis. The basis need not even be normalized, and, in that case, one replaces $\textbf{I} \textbf{I}^{\dagger}$ with $\textbf{E} \textbf{E}^{-1}$, where $\mathbf{E} \equiv \epsilon \mathbf{I}$ encapsulates $\mathbf{I}$, in its usual definition but where the basis vectors do not necessarily anti-commute. The scalar $\epsilon$ is then the product of all the lengths of the basis vectors. Finally, the inverse, $\mathbf{E}^{-1}$, is similar to a reversion but scaled down by that factor, $\epsilon$, such that $\textbf{E} \textbf{E}^{-1} = 1$. The proof is more rigorous 

% one can see that this would reduce to the special case I was discussing where $\mathbf{E}^{-1} = \mathbf{I}^\dagger$.

% % Simply replace $\textbf{I} \textbf{I}^\dagger$ with $\textbf{I} \textbf{I}^{-1}$, where 
% % \begin{equation} \label{eq:inverse_pseudoscalar}
% %      \textbf{I}^{-1} \equiv \frac{\textbf{I}^\dagger }{ \textbf{I}^\dagger \textbf{I} }.
% % \end{equation}



% Let us try to find $\vec{\gamma^2}$ using the procedure from section \ref{sec:duality}, equation \eqref{eq:dual_generator}, noting the result of the inner product.
% \begin{align*}
%     \vec{\gamma^2} &= (-1)^{2 - 1} \vec{\gamma_0} \wedge \vec{\gamma_1} \wedge \vec{\gamma_3} (\textbf{I}^\dagger) \\
%     \vec{\gamma^2} &= - \vec{\gamma_0} \vec{\gamma_1} \color{teal} \vec{\gamma_3}  (\vec{\gamma_3} \color{black} \vec{\gamma_2} \vec{\gamma_1} \vec{\gamma_0})  \\
%     \vec{\gamma^2} &= - \vec{\gamma_0} \vec{\gamma_1} \vec{\gamma_2} \vec{\gamma_1} \vec{\gamma_0}  \\
%     \vec{\gamma^2} &=  \vec{\gamma_0} \vec{\gamma_2} \color{teal} \vec{\gamma_1} \vec{\gamma_1} \color{black} \vec{\gamma_0}  \\
%     \vec{\gamma^2} &=  -\vec{\gamma_2} \color{teal} \vec{\gamma_0}  \color{black} \vec{\gamma_0},
% \end{align*}
% or
% $$\boxed{\vec{\gamma^2} =  -\vec{\gamma_2} }$$





\end{document}

