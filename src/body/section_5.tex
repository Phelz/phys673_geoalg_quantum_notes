

\section{Geometry of Spin}

\subsection{Rotations}
In the last section, I discussed the Pauli algebra, and how it arises from the bivector algebra within SGA. I would like to expand on this here, by first making a correspondence with a familiar operator. Consider
\begin{equation} \label{eq:euler_quantum}
	\large  \exp{i \hat{\sigma_z} \theta / 2} = \cos(\theta/2) + i \hat{\sigma_z} \sin(\theta/2),
\end{equation}
where $\hat{\sigma_z} $ is the quantum operator (sigma matrix) analog to our $\vec{\sigma_3}$. Equation \eqref{eq:euler_quantum} is relatively known in quantum mechanics, and can be verified by expanding the exponential function. The proof is simple and can be found in many textbooks. I would like to make the argument that the GA analog also holds, namely
\begin{equation} \label{eq:euler_ga}
	\large  \exp{ \mathbb{I} \vec{\sigma_3} \theta / 2} = \cos(\theta/2) + \mathbb{I} \vec{\sigma_3} \sin(\theta/2).
\end{equation}
Instead of going through the math, I would like to remind the reader of the fundamental result we stumbled upon while deriving the dual in equation \eqref{eq:dual_3d_test_1}:
\begin{equation} \label{eq:imaginary_pseudoscalar}
	\mathbb{I}  \mathbb{I} = \mathbb{I} ^2 = -1,
\end{equation}
which I swept under rug right momentarily. This result alludes to a hidden complex structure embedded within GA. Equation \eqref{eq:imaginary_pseudoscalar} holds not only for the pseudoscalar of 3D space which was used in deriving equation \eqref{eq:dual_3d_test_1}, but also for the 4D Minkowski space (as one quickly check), and, by extension, for the even subalgebra. This result is precisely why GA
\\ \\
Additionally, I have shown in the previous section that the Pauli vectors are unitary (see equation \eqref{eq:pauli_unitary}). Hence, we have all the ingredients necessary to assert that if equation \eqref{eq:euler_quantum} holds equation \eqref{eq:euler_ga} must also hold.
\\ \\
Finally, using equation \eqref{eq:pauli_product}, I would like to rewrite  $\mathbb{I} \vec{\sigma_3}$.
$$ \vec{\sigma_i} \vec{\sigma_j} = \epsilon_{ijk} \mathbb{I} \vec{\sigma_k} = \epsilon_{ij3} \mathbb{I} \vec{\sigma_3},$$
which implies,
$$ \mathbb{I} \vec{\sigma_3 } = \vec{\sigma_1 } \vec{\sigma_2 }. $$
Rewriting equation \eqref{eq:euler_ga}, we have
\begin{equation} \label{eq:euler_ga_rewritten}
	\large  \exp{ \mathbb{I} \vec{\sigma_3} \theta / 2} = \cos(\theta/2) + \vec{\sigma_1} \vec{\sigma_2} \sin(\theta/2).
\end{equation}
Here, I would like to examine the sandwich product
\begin{equation} \label{eq:sandwich_product}
	\large  \exp{ -\mathbb{I} \vec{\sigma_3}  \theta / 2} \ \vec{v}  \
	\exp{ \mathbb{I} \vec{\sigma_3} \theta / 2},
\end{equation}
for some vector $\vec{v}$ in our subalgebra. To illustrate, let us use $\vec{v} =  \vec{\sigma_1}$:
% \begin{align*}
% 	\exp{ -\mathbb{I} \vec{\sigma_3} \theta / 2} \ \vec{\sigma_1}  \
% 	\exp{ \mathbb{I} \vec{\sigma_3} \theta / 2} & = \left(  \cos(\frac{\theta}{2}) - \vec{\sigma_1} \vec{\sigma_2} \sin(\frac{\theta}{2}) \right) \vec{\sigma_1} \left(  \cos(\frac{\theta}{2}) + \vec  {\sigma_1} \vec{\sigma_2} \sin(\frac{\theta}{2}) \right) \\
% 	                                            & = \left(  \vec{\sigma_1} \cos(\frac{\theta}{2}) + \vec{\sigma_2}  \sin(\frac{\theta}{2}) \right) \left(  \cos(\frac{\theta}{2}) + \vec{\sigma_1} \vec{\sigma_2} \sin(\frac{\theta}{2}) \right) \\
% 	                                            & =  \vec{\sigma_1} \cos[2](\frac{\theta}{2})  + \vec{\sigma_2}  \sin(\frac{\theta}{2}) \cos(\frac{\theta}{2}) + \color{teal}\vec{\sigma_1} \vec{\sigma_1} \color{black} \vec{\sigma_2} \cos(\frac{\theta}{2})  \sin(\frac{\theta}{2}) + \vec{\sigma_2} \vec{\sigma_1} \vec{\sigma_2} \sin(\frac{\theta}{2})  \sin(\frac{\theta}{2}) \\
% 	                                            & =  \vec{\sigma_1} \cos[2](\frac{\theta}{2})+ \vec{\sigma_2}  \sin(\frac{\theta}{2}) \cos(\frac{\theta}{2}) + \vec{\sigma_2} \cos(\frac{\theta}{2})  \sin(\frac{\theta}{2}) - \color{teal} \vec{\sigma_2} \vec{\sigma_2} \color{black} \vec{\sigma_1} \sin[2](\frac{\theta}{2})   \\
% 	                                            & =  \vec{\sigma_1} \cos[2](\frac{\theta}{2})+ 2 \vec{\sigma_2}  \sin(\frac{\theta}{2}) \cos(\frac{\theta}{2}) - \vec{\sigma_1} \sin[2](\frac{\theta}{2})
% \end{align*}
% \begin{align*}
% 	\exp{ -\mathbb{I} \vec{\sigma_3} \theta / 2} \ \vec{\sigma_1}  \
% 	\exp{ \mathbb{I} \vec{\sigma_3} \theta / 2}  = \left(  \cos(\frac{\theta}{2}) - \vec{\sigma_1} \vec{\sigma_2} \sin(\frac{\theta}{2}) \right) \vec{\sigma_1} \left(  \cos(\frac{\theta}{2}) + \vec  {\sigma_1} \vec{\sigma_2} \sin(\frac{\theta}{2}) \right) \\
%                                                  =   \cos(\frac{\theta}{2}) \vec{\sigma_1} \cos(\frac{\theta}{2}) 
%                                                   - \vec{\sigma_1} \vec{\sigma_2} \sin(\frac{\theta}{2})  \ \vec{\sigma_1} \cos(\frac{\theta}{2})  \\
%                                                   &\quad + \cos(\frac{\theta}{2}) \vec{\sigma_1}  \vec{\sigma_1} \vec{\sigma_2} \sin(\frac{\theta}{2}) 
%                                                   -  \vec  {\sigma_1} \vec{\sigma_2} \sin(\frac{\theta}{2})  \vec  {\sigma_1} \vec{\sigma_2} \sin(\frac{\theta}{2}) \\
%                                                   % --------------------
%                                                  =  \vec{\sigma_1}   \cos[2](\frac{\theta}{2}) 
%                                                   - \vec{\sigma_1} \vec{\sigma_2}  \ \vec{\sigma_1} \sin(\frac{\theta}{2})  \cos(\frac{\theta}{2})  \\
%                                                   &\quad + \color{teal} \vec{\sigma_1}  \vec{\sigma_1}  \color{black} \vec{\sigma_2} \sin(\frac{\theta}{2})  \cos(\frac{\theta}{2})
%                                                   -  \vec  {\sigma_1} \vec{\sigma_2} \vec{\sigma_1} \vec{\sigma_2}    \sin[2](\frac{\theta}{2}) \\
%                                                   % --------------------
%                                                  =  \vec{\sigma_1}   \cos[2](\frac{\theta}{2}) 
%                                                   + \vec{\sigma_2} \color{teal} \vec{\sigma_1}  \ \vec{\sigma_1} \color{black} \sin(\frac{\theta}{2})  \cos(\frac{\theta}{2})  \\
%                                                   &\quad + \vec{\sigma_2} \sin(\frac{\theta}{2})  \cos(\frac{\theta}{2})
%                                                   -  (\vec  {\sigma_1} \vec{\sigma_2} \vec{\sigma_1} \vec{\sigma_2})    \sin[2](\frac{\theta}{2}) \\
% \end{align*}

$$ 	\exp{ -\mathbb{I} \vec{\sigma_3} \theta / 2} \ \vec{\sigma_1}  \
\exp{ \mathbb{I} \vec{\sigma_3} \theta / 2}  = \left(  \cos(\frac{\theta}{2}) - \vec{\sigma_1} \vec{\sigma_2} \sin(\frac{\theta}{2}) \right) \vec{\sigma_1} \left(  \cos(\frac{\theta}{2}) + \vec  {\sigma_1} \vec{\sigma_2} \sin(\frac{\theta}{2}) \right) $$
\begin{align*} 
                                                 &=   \cos(\frac{\theta}{2}) \vec{\sigma_1} \cos(\frac{\theta}{2}) 
                                                  - \vec{\sigma_1} \vec{\sigma_2} \sin(\frac{\theta}{2})  \ \vec{\sigma_1} \cos(\frac{\theta}{2})
                                                  + \cos(\frac{\theta}{2}) \vec{\sigma_1}  \vec{\sigma_1} \vec{\sigma_2} \sin(\frac{\theta}{2}) 
                                                  -  \vec  {\sigma_1} \vec{\sigma_2} \sin(\frac{\theta}{2}) \vec{\sigma_1}  \vec{\sigma_1} \vec{\sigma_2} \sin(\frac{\theta}{2}) \\
                                                  % --------------------
                                                 &=  \vec{\sigma_1}   \cos[2](\frac{\theta}{2}) 
                                                  - \vec{\sigma_1} \vec{\sigma_2}  \ \vec{\sigma_1} \sin(\frac{\theta}{2})  \cos(\frac{\theta}{2}) 
                                                  + \color{teal} \vec{\sigma_1}  \vec{\sigma_1}  \color{black} \vec{\sigma_2} \sin(\frac{\theta}{2})  \cos(\frac{\theta}{2})
                                                  -  \vec  {\sigma_1} \vec{\sigma_2} \vec{\sigma_1} \vec{\sigma_1} \vec{\sigma_2}    \sin[2](\frac{\theta}{2}) \\
                                                  % --------------------
                                                 &=  \vec{\sigma_1}   \cos[2](\frac{\theta}{2}) 
                                                  + \vec{\sigma_2} \color{teal} \vec{\sigma_1}  \ \vec{\sigma_1} \color{black} \sin(\frac{\theta}{2})  \cos(\frac{\theta}{2})
                                                  + \vec{\sigma_2} \sin(\frac{\theta}{2})  \cos(\frac{\theta}{2})
                                                  -  \vec  {\sigma_1} \vec{\sigma_1} \vec{\sigma_2} \vec{\sigma_1} \vec{\sigma_2}    \sin[2](\frac{\theta}{2}) \\
                                                  % --------------------
                                                 &=  \vec{\sigma_1}   \cos[2](\frac{\theta}{2}) 
                                                 + 2 \vec{\sigma_2} \sin(\frac{\theta}{2})  \cos(\frac{\theta}{2})
                                                 - \vec{\sigma_1}  (\vec  {\sigma_1} \vec{\sigma_2} )^2    \sin[2](\frac{\theta}{2})
\end{align*}
or,
\begin{equation} \label{eq:sigma_x_rotated}
     \large \exp{ -\mathbb{I} \vec{\sigma_3} \theta / 2} \ \vec{\sigma_1}  \
     \exp{ \mathbb{I} \vec{\sigma_z} \theta / 2} =  \vec{\sigma_1} \cos(\theta) + \vec{\sigma_2} \sin(\theta),
\end{equation}
where, in the last step, I made use of the half-angle trigonometric identities and that $ (\vec{\sigma_1} \vec{\sigma_2} )^2 = \vec{\sigma_1} \vec{\sigma_2} \vec{\sigma_1} \vec{\sigma_2} = - \vec{\sigma_1} \color{teal} \vec{\sigma_2} \vec{\sigma_2} \color{black} \vec{\sigma_1} = -1 $, again, behaving like the imaginary unit. The latter is no coincidence. In fact, the Pauli algebra is isomorphic to the algebra of quaternions, where 
\begin{align*}
    \vec{\sigma_1} \mathbb{I} &\sim i &
    -\vec{\sigma_2} \mathbb{I} & \sim j &
    \vec{\sigma_3} \mathbb{I} &\sim k.
\end{align*}    
Hence, the quaternion algebra, fundamentally related to spatial rotations, is also a subalgebra of the SGA, and the imaginary unit $i$ is represented by the bivector $\vec{\sigma_1} \vec{\sigma_2}$. Equation \eqref{eq:sigma_x_rotated} describes a rotation of the vector $\vec{\sigma_1}$ about the $\vec{\sigma_3}$ axis by an angle $\theta$. Now we can visualize the rotation as being brought about by the action of the exponential of the bivector $\vec{\sigma_1} \vec{\sigma_2}$, which is a rotation in the plane defined by the two vectors $\vec{\sigma_1}$ and $\vec{\sigma_2}$. The sense of rotation would then be opposite if we use the bivector $\vec{\sigma_2} \vec{\sigma_1} =-  \vec{\sigma_1} \vec{\sigma_2}$, a purely geometric intuition! 
\\ \\ 
It is because of this anti-commutativity that we can write 
\begin{equation} \label{eq:rotor_form}
    \large R_{\vec{\sigma_3}}(\theta) =  \exp{ -\mathbb{I} \vec{\sigma_3} \theta / 2},
\end{equation}
and,
\begin{equation} \label{eq:rotor_form_reversed}
    \large R_{\vec{\sigma_3}}(\theta)^\dagger =  \exp{ \mathbb{I} \vec{\sigma_3} \theta / 2},
\end{equation}
since a swap (or reversion, denoted by the dagger) of $\vec{\sigma}_1$ and $\vec{\sigma}_2$ in the exponent of equation \eqref{eq:rotor_form} gives a minus sign. Finally, one can check that this operation is indeed a rotation. For instance, substituting $\theta = \pi/2$ in equation \eqref{eq:sigma_x_rotated} gives
$$ R_{\vec{\sigma_3}} \left( \frac{\pi}{4} \right) \ \vec{\sigma_1}  \ R_{\vec{\sigma_3}} \left(  \frac{\pi}{4} \right)^\dagger =  \vec{\sigma_2}. $$
Substituting $\theta = \pi/4$ gives
$$ R_{\vec{\sigma_3}} \left( \frac{\pi}{2} \right) \ \vec{\sigma_1}  \ R_{\vec{\sigma_3}} \left(  \frac{\pi}{2} \right)^\dagger =  \frac{1}{\sqrt{2}} \left( \vec{\sigma_1} + \vec{\sigma_2} \right). $$
Indeed, we are moving around the unit circle in the plane defined by $\vec{\sigma_1}$ and $\vec{\sigma_2}$! Thus, using these rotations, we can move around the unit sphere, commonly referred to as see Figure \ref{fig:riemann_sphere}. No surprise, this unit sphere corresponds with the famous Bloch sphere, popularized by Bloch in Nuclear Magnetic Resonance, or the Poincaré sphere, more appropriate in the context of quantum optics, used to describe the polarization of light. However, this mathematical entity should be rightly attributed to Riemann, who first introduced it in the context of complex geometry. In the next section, I shall explore how this sphere relates to the mathematical description of the qubit.

\begin{figure}[H]
   \centering
   \includegraphics[width=1\textwidth]{figures/rotations_merged_all_2.png}
   \caption{ (a) a rotations of $\theta = \pi/4$ about $\vec{\sigma_3}$, (b) a rotation of $\theta = \pi/2$ about $\vec{\sigma_3}$, and (c) how arbitrary rotations can allow us to move around the unit sphere.}
   \label{fig:riemann_sphere}
\end{figure}


\subsection{The Riemann Sphere}

\subsection{The Qubit}

In quantum mechanics, we would associate a state to the vector shown in Figure \ref{fig:riemann_sphere}(c). One could write this arbitrary state as
\begin{equation} \label{eq:qubit_state}
    \ket{\nearrow} = v \ket{\uparrow} + w \ket{\downarrow},
\end{equation}
where $\ket{\uparrow}$ and $\ket{\downarrow}$ are the basis states, and $v$ and $w$ are complex numbers. The latter implies we need 4 real numbers to describe the state, which would more appropriatly correspond with two spheres! The correspondence, therefore, is subtle, and relies, most importantly on the idea of global phase invariance. Namely, if we scale the state by a complex number $\lambda = e^{i \phi}$, the state is unchanged:
\begin{equation} \label{eq:global_phase}
    \lambda \ket{\nearrow} =  \ket{\nearrow} .
\end{equation}
This allows us to scale the state by $\frac{1}{v}$ such that 
\begin{equation} \label{eq:qubit_state_normalized}
    \frac{1}{v} \ket{\nearrow} = \ket{\uparrow} + u \ket{\downarrow},
\end{equation}
where $u = \frac{w}{v}$. 




